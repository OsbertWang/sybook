% !TeX root=../main.tex
\chapter{破庙}
火光四起,周围的空气弥漫着一股浓烈的血腥味而夹杂着一些刺鼻焦肉味,只见师父镇定自若的看着地上一具具被烧焦的尸体,走了过来。

“好徒弟,带着阆风剑,去京城找你师伯,万不可让它落入歹人之手。往后师父不能罩着你,全看你自己了!”

“师父……师父!”

江风满头大汗地惊醒,半晌没回过神来。

雨已经下了一天一夜,江风栖身的山间野庙早就破了顶,地上坑坑洼洼,满是积水。还好,倾斜的佛像和大梁支撑起了一个还算干爽的地方,江风躲到破庙之后,便寻着了这处容身之所。

为了避免麻烦,江风没有生火,从梦中惊醒之后,夜似乎更凉了一些,寒气透骨而入,眼下竟觉得身体僵硬。他挣扎着动了动,身上大小伤口瞬间被尽数牵动,疼得他撕心裂肺。

江风紧了紧绑着阆风剑的布带,沉默地用自己的配剑撑地,咬着牙试图站起来。

正在这时,远处突然传来细密的脚步声。

它们混杂在虫鸣中几不可闻,幸而江风耳力过人,还是分辨了出来。他立刻警戒起来,深吸一口气,硬撑着移到神像之后的阴影里,心底却十分清楚,在破庙这丁点儿大的地方,他一定躲不过对方认真的搜查。

江风凝神听去,对方似乎轻功不高,他们踩在地上,大大小小的水坑都被踩得噗噗响。没过多久,他们的说话声就传入了江风的耳朵。

“大哥,那阆风叛徒真在这破庙里面?”

“线报说他往西去了,大晚上的,下这么大雨,他又受了伤,一定在破庙!”

“线报只说他受伤,未必伤得很重呀……这厮可屠灭了整个阆风台,还夺走了阆风剑,万一他只是轻伤,我们几个普通人送上门……”

他的话还没说完,领头的人大声喝道:“放屁!你以为阆风台上尽是你这种废物么!那阆风台上武功高强的真人贤者不知多少,这江风就算练就邪魔功夫,再强他也只是个凡人!屠灭整个阆风台还想全身而退?!”

他们的交谈还在继续,江风心中已是巨浪滔天,苍白的脸上最后一丝血色也褪去了——不仅师父去世,阆风台满门被灭,凶手竟是他江风?!

若这是噩梦一场,尚且荒谬惊悚,何况现实。

冰冷的雨水滑下瓦片,滴到他衣襟,触到了江风的伤口,而江风恍若未觉。

那几人还在交谈,带头的却颇为感慨地说道:“想我平日也与这江风打过几次交道,端的一副正人君子、生人勿进的模样,要不是明震道长,险些叫他这魔头逍遥法外了!”

那小卒立刻附和道:“可不是么,多亏明震道长聪明,装死活了下来,要不是他出来指证,谁也不知道这阆风台竟出了这等丧尽天良的畜生!”

说话间,他们一队人马已经走到了破庙近处。庙中,江风颓然依靠断柱,心中阵阵发寒,整个人从头凉到了脚。

“明震……明震……”

江风自幼被无为真人收为首徒,年纪不大,辈分却高,同门中年纪相仿的大多要称他一声师叔。平日里,江风为人稍显冷淡,做事又极讲规矩,阆风中人大多敬而远之,更妄谈亲近了。这四师叔的徒弟明震是少数几个同江风走得近的人,或者说,是最亲近的。

江风一时分不清自己是怒是悲,一口气堵在胸口,喉间竟泛起一股腥锈味道,待反应过来,他已哇地吐出一口血来!

“老、老大!庙、庙里好像有动静!”

“慌什么!来啊,给我围起来!一只老鼠也不许放出去!”

那伙人已经到了近前,这一声令下,数十支火把熊熊燃起,破庙外恍若白昼。

“江风!你已插翅难逃,还不快束手就擒!”

难道今日真的要葬身于此?可阆风剑决不能落入他人手中,为了它,师父已经亡故,难道我连它也护不住……

江风握紧佩剑剑柄,随时准备拔剑,可外面突然传来了惊恐的叫声,外面的人全然不复方才的笃定,失魂落魄似的喊出了他的名字。

江风偷偷瞄了一眼,庙外一行人共二十有余,乃是阆风台山脚下,仙沢城的差役。领头的捕头从前还打过交道,这时他仿佛壮着胆子大声喊道:“江风!你自幼拜在国教阆风台门下,为何私练邪功?被人撞破后,竟屠戮众多同门弟子!多亏明震道长侥幸逃脱,将你的罪行一五一十说了出来!你这欺师灭祖之徒,杀了同门还不够,竟然连山上的客卿都下了狠手!官府已经下了海捕文书,知县大人下令封锁城门,并令我等在城外搜索,你已经插翅难飞了!”

这明震同江风结伴下山办事已有多次,大家都知道他二人素来和睦,因此明震所言并未引起怀疑。更何况,阆风台贵为国教,每代必出一德高望重者坐镇帝都,承国师之位,眼下这等惨状,怕是已上达天听。任何人只要捉住凶手,即便不是重重有赏,也可前途无量。捕头急功近利,才得了线索,就私下带了兄弟来抓人。他以为江风重伤,自信可以将其制服。

然而此时,他们一行二十余人,皆望着屋顶一角上那一袭黑袍,吓得瞪大了眼睛——在二十几双眼睛的注视下,这人就悄无声息地立在了那摇摇欲坠六角飞檐上,彷佛凭空出现。

那捕头率先回神,强自镇定,朝房顶上的人影道:“来者可是江风?”

黑袍人纹丝不动,只有衣袂在风中猎猎作响,弯月立于脑后,更显神秘莫测。

捕头又道:“江风!你已被我们包围,还不快束手就擒!”

对方仍是不答,却从黑袍下拿出了剑,瞬息之间,宝剑如龙吟出鞘,被月光映出一片雪亮,晃花了人眼。差役们反应不及,只见那黑袍人飞身而下,一剑刺来——静时若处子,动时若脱兔,哪有一丝一毫“重伤”的样子?

捕头立刻抽刀,挡住了这一剑。说来也怪,这黑袍人身法虽快,剑势却不甚刚猛。捕头刚松一口气,只听“咔嚓”一声脆响,刀竟应声断成了两截——断刀切口整齐,可见并非裂开,而是被剑刃在瞬息间切了开来——黑袍人彷佛没料到手中的剑竟有削铁如泥的程度,一瞬间怔在了原地,那捕头哪想得了这许多,连滚带爬跑开了,还不忘回头对着黑袍人吼道:“你们还愣着干什么?那把剑一定是阆风剑!他就是江风!大人有令!捉住江风者,赏黄金百两,官升三级!”

虽有重赏在前,可差役们已见识到那柄宝剑的威力,哪敢妄动?黑袍人身边的包围圈越来越松——若他们此时肯稳一稳心神,当会发现,眼前的对手相比方才高立房顶之时,身量矮了许多,帽子更是晃晃荡荡——可差役们这时候被吓破了胆,黑袍人亦是不肯给他们反应的机会,提剑与他们战到了一处。差役们瞅准空档,失魂落魄地弃刀逃向树林深处。转瞬间破庙外又变得冷冷清清,掉在地上的火把也只冒着青烟。

江风终于支撑不住,背靠着倾斜的神像滑了下去,又把胸口郁积的气都吐了出去。那黑衣人所用招式确为阆风剑法,但功力欠缺,只是仰仗着身法飘渺,用剑招晃得人眼花,而他所使的轻功又确是“风过无痕”,这让江风想起一个人来,正待深思,一个脚步声传来,惊得江风猛抬起头。

“江道长,是你吧。”

“谁!”

“江道长不记得我了?”女子的声音颇带着几份娇俏。她点着了从庙外捡起的火把,照亮了破庙,也照亮了她——那是个十五六岁的姑娘,身材高挑,穿着做工考究的红色锦缎小袄。往上看,这姑娘脸上沾了些灰尘,颇有些狼狈,却掩不住其容貌姣好,一双杏眼彷佛装进了漫天星光。

“怎么是你?”江风声沉如水,听不出半点喜悦,防范之意不言而喻。

江风当然认识她,扶英剑庄的二小姐,叶安苒。数日前,扶英剑庄的人马押送定制的刀剑来阆风台做客,正是江风负责接待这个在江湖上极负盛名的叶二小姐。

人在江湖上有名,无外乎两种原因,武功高强亦或行事作风特别,多的是两者兼得,可叶安苒在江湖上出名,同这两件事没有一丝一毫的关系,而与一个离谱得有些可笑的事情有关——她不会武功。

此事盛传武林,连从来对江湖纠纷漠不关心的江风都知道一些传言,他认识叶安苒后,便知道传言不虚,叶安苒确实不会武功,说是手无缚鸡之力都不为过。

然而一个全然不会武功的女子,却从变为人间炼狱一样的阆风台上全须全尾地活了下来,身上虽然狼狈却一丝伤都没有,这几乎是不可能的。

联想到方才得知的消息,让江风如何不戒备这叶二小姐?

叶安苒看江风丝毫没有走过去的意思,又用火把周围照了照,待看清了地上的障碍物分布,她边向江风走去边道:“这里不是说话的地方,我先扶你走,万一他们回来……你干什么?!”

长剑泛着寒光横在二人之间,剑柄握在江风手中,剑尖抵在叶安苒的胸口,火红的锦缎被划破,露出其中的棉絮。

江风的眼眸,比剑光更寒。

此时,仙沢城内,万籁俱寂,只剩更夫敲着梆子,喊出一声声的“天干物燥——小心火烛——”。在更夫身后的房顶上,一个黑影一闪,进了客栈唯一一扇开着的窗户。

“事情办妥了吗?”问话的人隐在阴影中,看不清面容,声音低沉沙哑,令人听了不快。

黑影单膝跪地,毕恭毕敬应道:“回禀堂主,属下等没有在阆风台找到那样东西。想必,是在江风身上。”

“哼,一群废物!”那堂主一声低喝,黑影的头立刻压得更低。

“看来本座得去请示上峰的意思。你带人继续寻找江风,活要见人,死要见尸。记住,一定要找到那样东西!”

“属下遵命。”