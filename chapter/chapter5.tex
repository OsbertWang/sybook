% !TeX root=../book.tex
\chapter{练武}
江风还未曾反应过来问话的对象,陆秋初便弱弱说了一句:“是我……从掌门房里……偷的……”

唐玖言看向陆秋初,目光如剑,扎得陆秋初浑身不自在。

“因为小秋的剑法一直练不好,所以她就仗着自己的身份,拿了阆风剑,独自去林子里了。”叶安苒接过话头,抢先说了一句。

陆秋初听完,也顺着叶安苒的话说:“我……我想用阆风剑来完成执教师兄留下的课业……它既然是祖师爷的佩剑,那威力应该比我的剑好多了。只要我把课业完成,师兄就不会再训我了!”

江风突然厉声说道:“小秋!你可知盗取掌门信物,该受何等惩罚?”

只见陆秋初眼中转泪,不再言语;而叶安苒也一时也不好再说些什么。

“叶二小姐果然冰雪聪明。”一直盯着陆秋初的唐玖言突然出声,说完,他看了一眼叶安苒,才起身对江风说道,“江少侠,你们在此多待几日吧。”

随即,唐玖言让唐栗带他们三人离开。待三人走后,唐玖言一个响指,对着四下无人的房间说道:“让扶英剑庄来接人。”

回到房间,江风三人依旧心情沉重。

陆秋初进了门便一直低着头,而江风暗暗叹了口气,反倒是叶安苒把手搭在陆秋初的肩膀上说道:“你方才做得很好。”

“爹爹……还有师伯他们……真的、真的……”

陆秋初说到这里再也说不下去,捂着嘴,埋进叶安苒怀里。叶安苒红了眼眶,只是紧紧抱着她。

江风看着这一切,只说了句“哭出来,会好受些”,便离开房间。

站在院内,江风望着空中新月,只觉身心俱疲。

自那日之后,江风三人一直在别苑静养,却始终不见唐玖言的踪影,每日三餐皆由唐栗亲自配送。三人除却不许离开别苑外,一切行动都不受限制。

江风每日在房内打坐冥思,多次尝试运功,希望能将体内的毒逼出去,最终却只是徒劳。

叶安苒除却在院内闲逛,便时常来找江风聊天,有时还替江风把脉以了解江风中毒的情况。

陆秋初深知自己的武功差距。她不愿意再拖累江风,但自己练习却难有长进。思忖再三,她找来唐栗与自己练习。

后院当中,唐栗立起梅花桩。一个跨步,他跃上桩子,对着陆秋初招招手,说道:“陆小姑娘,既然要我陪你练功,那咱们先约法三章。第一,你我只在这桩上比试。第二嘛——接着!”

说罢,唐栗扔给陆秋初一把木剑。陆秋初仔细端详这把木剑,发觉它比自己的佩剑略长一些,剑的前端用白布裹着,已经蘸满了白面。

“这第二条,你若能在我这藏青色衣袍上留下一个白点,就算你赢。”

陆秋初紧抿双唇,思考再三,问道:“那第三条呢?”

唐栗笑道:“你先上来,我再与你说这第三条。”

陆秋初听罢,用起本门轻功,飞上梅花桩,与唐栗正面相对。

到了桩上,陆秋初才明白唐栗为何一定要选这梅花桩。

这些桩子看似稳定,可只要人一站上去,桩体就会抖动。陆秋初稳住自己的下盘已是不易,而唐栗倒是在桩上如履平地。 

唐栗看着陆秋初勉强才能维持平衡,哑然一笑,说道:“听好第三点。你我比试,你可以喊停。但只要你喊停,日后就不许再来找我。你师兄功夫应该不错,不找我还可以找他嘛。”

陆秋初皱紧眉头,对着唐栗喊道:“我才不会喊呢!”说完,就准备攻向唐栗。不料她刚摆出起手式,梅花桩又抖动了起来。

唐栗挑起一边眉毛,对着陆秋初说道:“你这小妮子倒挺倔。那好吧,今天我再让你一点,若我双手有一丝活动,也算你赢。”

“看剑!”陆秋初听罢,眼中放光,提剑就向唐栗而去。

可惜,十招不到,陆秋初就被唐栗制服了。

“气势不错,可惜功夫太差。再来。”唐栗收招,跳出陆秋初三尺之外。

江风在屋内听到打斗之声,便打算到后院看看。一方面他担忧师妹,另一方面也好奇唐栗的功夫。

走到后院时,江风看到叶安苒已在檐下伫立。他走到叶安苒身边,静静看着院内发生的一切。

“太慢了!”唐栗迎面对着刺来的剑,不慌不忙地避开了即将触到自己的剑锋。而后,他抬腿直向陆秋初面门而去。陆秋初回剑来挡,却被这一脚的力道震下了桩子。

“第八次了,还打吗?”站在桩边,唐栗俯视着狼狈的陆秋初,饶有兴味。

陆秋初气喘吁吁,双目直盯着唐栗。

在一旁观战的叶安苒笑着对江风说:“看来她真是专心,连你来了都不知道。”

江风点了点头,回应道:“倘若她知道我在,怕是又要分心。”

“真想不到小秋竟然会找唐栗小哥。更没想到他竟会答应。”叶安苒一边看着两人,一边与江风闲聊。

“是啊。许久不见她如此认真,倒真是令人欣慰。”江风紧盯着唐栗,脑中一直思考着如何化解他的招数,仿佛与唐栗切磋的是他本人一般。

陆秋初持剑向上斜挑,被唐栗侧身闪过;接着她转为平削,妄图划过唐栗前胸。本以为唐栗不可能再躲,哪想他身法迅疾,竟瞬间闪到陆秋初背后,一脚袭来。

陆秋初随即下腰躲闪,转身横劈唐栗下三路。唐栗见状及时收招,后撤两步又闪开了。

叶安苒低呼一声,“可惜,小秋能打到他就算赢了!”

江风摇摇头:“唐栗比小秋快上太多,若是实战,小秋的剑还未出招,唐栗已伤到她了。”

叶安苒瞄了江风一眼,道:“江道长,小秋才十四岁啊。”

江风并没有接着叶安苒的话,反而说道:“她的资质远胜许多同辈弟子,可惜机巧有余,勤奋不足。是以剑术修为才迟迟没有更进一步。”

“小孩子嘛,聪明就难免不太努力。要是再天赋异禀,可能就有些心高气傲了。”叶安苒想起家里某个不省心的少年,止不住露出笑容来。

“好了,到此为止。”唐栗收招,对着陆秋初笑了笑,将一直靠在背后的双手活动了一下。

“我、我还可以的,再陪我练几局好不好!”就在说话的同时,陆秋初的脸上已经滴下了汗珠。

“我可不像你这么闲。再练下去,我家公子就要不高兴了。”

听他说完,陆秋初也不好强求,便与唐栗一同下了梅花桩。

唐栗离开后,陆秋初并未回屋休息,她一边想着刚才唐栗的出招,一边思索着本门的剑法。但还是不知道为什么自己的剑总是在即将打到唐栗时被躲开。

江风看着陆秋初,稍作思量,便折了一截树枝走过去。

“小秋,你再演练一遍方才对打时的情形。”

“师兄?”陆秋初没想到江风会来,愣了一下,随即道:“好,那师兄小心了!”语毕,她便以一招平刺向江风袭来。

就在陆秋初即将打到江风时,江风一个侧身避开剑锋。陆秋初一惊,这与方才唐栗的身法一模一样。

未等陆秋初反应过来,江风已用树枝打到了陆秋初的太渊穴。陆秋初吃痛,手劲一松,把剑丢到了地上。

“出剑要准,不可有多余动作;持剑要稳,切不能丢下兵刃。这一招平刺,你速度尚可。但出招前,你的手臂有轻微摆动,只这一瞬,便足以让人判断出进攻角度。把剑捡起来。”

就这样,江风在全无内力的情况下与陆秋初对打十余回合,却将陆秋初的攻势尽数化解。

在一旁观看的叶安苒倒是兴致不减。这是她第一次见江风出手,虽然江风内力全无,但一招一式间却与陆秋初有着全然不同的气势,“真不愧是无为真人大弟子啊……”

半个时辰下来,陆秋初已见疲态,但比起与唐栗在梅花桩上对战时,她的出招已透出些灵巧利落,这样的进步着实让江风欣慰。

可惜即便如此,陆秋初仍然攻不破江风的严密防守。自二人交手以来,江风寸步未移,陆秋初却招式渐穷。

江风心知,今日再练下去也难有大的成效,还是应该让陆秋初仔细思考一下今日所学才是。

念及此处,江风止住了陆秋初。

“师兄?”

江风道:“习武切忌心浮气躁,你已练习颇久,不可过度劳累。量力而行,持之以恒,方为正途。”

陆秋初收敛心神,缓缓吐出一口气,笑着说:“知道了,谢谢师兄。”

这时,叶安苒也走了过来。

“安苒姐姐,你也在啊。”陆秋初有些惊讶地说道。

叶安苒看着陆秋初。她原本稚嫩的小脸上,汗水和沙尘已经凝聚在了一起,衣服上也被沙土包裹了一层,仿佛某处还有损坏。叶安苒一边笑,一边心疼。她抓起陆秋初的手,问道:“刚才掉下来,有没有伤到?咱们回屋吧。我给你洗漱一下,再换一身干净的衣服。”

“好!谢谢安苒姐姐。”说罢,两人一起回屋梳妆打扮去了。

江风见她二人回屋,便走到梅花桩旁。他敲了敲几根桩子,又用手摇了摇。这些桩子是唐栗今日才做好的,桩体实心,每根长短宽窄一致,但埋入地下的部分极窄,而上面供人行动的部分又极宽。

“这样的桩子,也真亏了小秋能坚持这么久……”