% !TeX root=../main.tex
\chapter{阆风}
待唐栗将江风一行人带到客房,直至傍晚,他们都一直没有唐玖言的任何消息,反而被允许在后院内自由行走。

陆秋初早已在车内待得闷了,本打算借此时机在后院内散散心,但一看到江风,便打消了这个念头,转而去院内练剑了。江风见此,心中宽慰。既然唐玖言不会对他们不利,江风自然也不用担心自家师妹的安危。可是,唐玖言所提的交易,尚且悬在心中。好在叶安苒已提出了应对之法,两人正好利用这短暂的平静考量各种细节。

入夜后不久,一阵敲门声打断了交谈。江风与叶安苒随即起身开门,只见是唐栗站在门前。

江风邀请他进屋稍坐,唐栗婉言谢绝,只说:“晚饭已备好,我家公子差我来请江少侠及二位小姐。”

在院内练剑的陆秋初看到屋内的情况,便急忙赶了回来,正好听到了唐栗的话。

见状,叶安苒抢话道:”还请唐小哥稍等,小秋刚练完剑,总要梳洗一下。“接着转而对江风说:“江道长,你也先出去等一下吧,一会儿就好。”说完,叶安苒便对江风使了眼色,江风点头,便带了剑与唐栗一同在廊内等候。

在廊内,唐栗看看江风,又看看外面的景色,一副欲言又止的模样。沉默许久后,唐栗终于忍不住开口道:“江少侠,不妨由我来拿这柄剑吧。”

江风答谢一句:“承蒙好意,只是此剑于我意义非凡,不好托于他人。反倒是你家公子,轻功了得,在下十分敬佩。据在下所知,江湖上唯有两个门派能使得如此轻功,不知你家公子师承何人?”

唐栗下意识地松了下自己的袖口,有些为难道:“江少侠,我家公子不喜属下谈论他的事,”随即又笑了起来,“更何况,您心中应该也有大概了吧。”

“在下心中确有猜测。”江风毫不掩饰,“只是在下好奇,为什么唐公子会对我阆风之事如此在意。”

这时,叶安苒带陆秋初从屋内走出,“我们好了,走吧。”

唐栗见状,便说:“那就请几位随我来吧。江少侠的问题,恐怕只有去问我家公子了。”

三人跟着唐栗到了大厅。大厅正中,一张红木圆桌端正摆着,旁边有四张椅子,唐玖言坐在正席,旁边的顾旻正附耳听着唐玖言的吩咐。三人见状,便在稍远的地方等待。只听唐玖言交待了句“去吧”,顾旻便走出了屋子。

接着,唐玖言起身请三人入席。见几人坐定,唐栗便召唤庖厨上菜,不多时,菜肴便占满了整张桌子。这其中,有从老林获取的山珍,有从深海捕捞的海味,更有就地采摘的果蔬。江风持重,叶安苒识广,面对此景倒也冷静,只是陆秋初小声惊叹了一句。

“招待不周,望江少侠不要见怪。”

“唐公子客气了。”

“请。”

语毕,四人围桌而食,只留唐栗一人在旁服侍。这时,江风才发现自己所饮并非是酒。正好奇时,唐栗上前为江风续杯,说道:“青玉露,我家公子的珍藏。”

江风想起在矮屋外的事,便知这青玉露定非凡品。

一餐饭后,唐玖言招呼江风一行人到旁边客厅稍坐,唐栗命仆役们上茶、撤下菜肴,又带着他们一同退了出去。

唐玖言一手支腮,一手把玩铃铛,说道:“江少侠,你我的交易,如何了?”

江风施礼回道:“实不相瞒,阆风台遭逢巨变之时,在下正在门派后山巡视。待发现门派内有异样,正欲回赶,却被黑衣人拦了去路。这些人配合紧密,出手狠毒,应是训练有素的刺客。几番交手,我与师弟们均受重伤,是以无法赶回门派,亦不曾亲眼得见门派内的情况。反而是叶小姐,竟能带着师妹从门派内逃出,在下钦佩。”

“江道长客气了。”叶安苒回道,“一路上若非有小秋帮忙,只怕我也会遭了那些黑衣人的毒手。”

唐玖言略一哂笑,看向叶安苒:“那,叶小姐可愿替江少侠完成这笔交易?”

叶安苒笑道:“唐公子,小女子有个不情之请。”

唐玖言不说话,叶安苒也猜到会如此,便接着说:“可否请您告知,究竟是何原因,唐门九公子会对阆风台如此好奇?”

唐玖言终于正坐起来,“这要看你说得如何。”

叶安苒心知,这唐玖言周身是迷,断不会轻易说出自己的目的,也就没再纠缠,慢慢说道:“我奉家父之命,上阆风台见无为真人。早些时候,阆风台曾向我家买了一批刀剑,我去交货,是以认识了江道长……”

话说当日,江风奉师命,招待叶家一行。而后,他便带着叶安苒在门派内行走。等到了藏书阁,巡山弟子来报,说后山异样。江风旋即决定亲自前去查看。叶安苒见江风忙碌,也就不再留他,只一人待在藏书阁。恰巧,陆秋初正在藏书阁翻阅典籍,因此结识了叶安苒。两人一见如故,便相约第二天去阆风林。

当夜,叶安苒睡意不深,朦胧中忽然听到有人大喊“走水”,她便穿衣出门查看。当时应是子时刚过,火龙肆虐,阆风众弟子疲于救火。叶安苒见状,便吩咐手下帮忙。

突然,不知从何处杀出一群黑衣人,他们不由分说,见人就杀。众弟子和门客们只顾救火,因此被打了个措手不及。对方人数众多,且有备而来,一时间阆风弟子竟是溃败。

当时,江风早已赶回门派,然而纵使他再强,终究双拳难敌四手,加之对方配合默契,他竟无法再前进一步。正交手间,他一个回身,虽避开了刺向要害的利刃,却还是被其划伤。再打下去,江风只觉内力越发空虚,神志也逐渐模糊,这才意识到,自己已经中毒。

直到七位长老亲率弟子布阵御敌,阆风台方止住了颓势,然而江风身边的敌人却有增无减。眼看大势已去,江风忽然听到熟悉的声音。

“伤我徒儿,也要先问问老道同不同意!”

无为真人从天而降,挡到江风面前,从他体内突然迸发出强大内力,将面前的黑衣人一扫而空。

江风见状,欲阻止师尊强行催动内力,却不想被师尊带着离开了门派,直奔阆风林。

密林深处,氤氲满布。无为真人带江风来到了一处山脚,摸索着周遭的机关。只听身后传来沙哑阴狠的声音:“无为,有些人你开罪不起。识相的话,把东西交出来,我或可保你阆风的徒子徒孙!”

无为真人不为所动,打开机关密道后,一掌把江风推了进去,只留给他一把剑和一句话。

“好徒弟,带着阆风剑去京城找你师伯,万不可让它落入歹人之手。往后师父不能罩着你,全看你自己了!”

江风本打算劝师尊与自己一同走,然而无为真人心意已定,待江风进入密道后,便关闭了石门,又损坏了机关。这石门重逾千斤,一旦关闭,任谁也不能将其打开。

江风无力地拍打石门,只能是徒劳无功。逃出密道之后,他不敢连累旁人,也怕黑衣人找到自己,只在认识的医馆取了伤药,但身上的毒却无法可解。

谁又能想到,逃出升天的江风却遭同门污蔑,官府四处张贴通缉布告,自己的行踪也就从医馆那边泄露了出去。

话说两头,当夜,叶家人无心恋战,只护着叶安苒一路后退。几位忠仆为保护她尽数死在黑衣人手下,只剩一位表叔,在浑身是伤的情况下,依然护着她逃到了阆风林。这时,她正巧遇到了陆秋初。

原来,白天陆秋初与叶安苒分别后,便前往阆风林。说来也怪,这阆风林的气候与周遭全然不同。它深入山谷之中,不受时令影响。这时节,层林尽染,煞是好看。陆秋初在林中穿梭,看似游玩,实际是在修炼“风过无痕”。练习累了,便在林中睡着了。等她醒来才看到门派中火光冲天,急忙往回赶。

叶安苒与陆秋初相遇后,只简略跟她说了门派情形。陆秋初心急如焚,可是又不能放下眼前的两人,一时不知该如何是好。这时,伤重的叶家表叔拜托陆秋初,一定要护送叶安苒回到叶家,便断气了。

叶安苒也顾不得悲伤,只问小秋,这林中哪里可以藏身,陆秋初便将其带到了密林阵法当中。此阵法原是无为真人结合奇门遁甲与阴阳八卦所创,休门连接密林,杜门连接山下村镇。陆秋初武艺不佳,然对奇门遁甲颇有天赋,同辈弟子当中,唯有她能走出这阵法。

“后来,小秋就带我走出了阵法,又遇到了江道长。后面的事,唐公子就都知道了。”叶安苒将自己所知的事情,一五一十地说了出来,但隐瞒了所有跟江风相关的细节。讲到表叔,她也是眉目悲伤。同样悲伤的,还有一直在旁的陆秋初。她倔强地咬着嘴唇,不肯让眼泪掉下来。

反观江风,倒是愤恨大于悲痛。他一边听着叶安苒的描述,一边抚摸着阆风剑——这世代相传的掌门信物。当夜之事,虽恍若隔世,却又无比清晰。他心想:“明震,我一直将你当成推心置腹的朋友,却不料竟遭你污蔑。他日再见,我定要你在师父灵前说出真相!”

正思想时,只听唐玖言问了一句:“阆风剑,你怎么拿到的?”