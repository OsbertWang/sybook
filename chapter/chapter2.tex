% !TeX root=../book.tex
\chapter{逃跑}
破庙中的僵持还在继续。此时的叶安苒能够明显感受到江风身上散发的浓郁杀气。

自己冒着生命危险来救人,对方非但不领情,还会拔剑相向,这样的场景,只怕谁都不会冷静。好在这位叶二小姐天生机敏,旋即明白了个中缘由,嗤笑一声,嗔道:“我如果要害你,还会一个人站到你面前吗?”

江风继续靠着半截梁柱撑住身体,手上的剑却纹丝不动。

叶安苒仔细打量了江风,面白唇浅,周身血污,呼吸急促,一副半死不活的样子。她无奈地说道:“本打算来救人的,可看现在这样,我是多事了。现在城里城外都是要抓你的人,你好自为之吧。”

说完,叶安苒转身要往外走,突然听到身后扑通一声,急忙回头,只见江风跪倒在地,全凭配剑支撑身体。

轻叹一声,叶安苒又走过去,扶着江风坐下,为他诊脉。江风虽不情愿,但体力早已不能支撑自己,连站立都做不到,又如何能够挣脱?

叶安苒看他终于略微配合了些,心中倒是一喜,略微得意地说:“你看你,这么凶巴巴的,阆风台就这样教你待人接物啊?当初在山上时还是一副温文尔雅的样子,怎么现在冷冰冰的?”叶安苒略微停顿,仿佛是在等着江风反驳,然而江风依旧紧闭双目,不言半语,叶安苒只好继续说道:“好啦,我知道你在想什么,一会儿等门外的人进来你就懂了,现在还是留着点力气吧,待会儿还要逃命呢!”

江风听完,双目缓缓打开,问道:“门外的人是谁?”

叶安苒并没有回答他的问题,脸色却突然凝重起来,她仔细看了看江风的伤口,用手指沾了一点血,轻轻摩擦过后,放在鼻尖小心地闻了闻,最后,她稍带迟疑地用舌尖沾了些手指上的血,又急忙吐了出去,眉头轻皱,低喃一句,“有毒……”

听罢,江风淡淡说道:“可惜不知道是什么毒。”

“此地不宜久留。”叶安苒起身,架起江风的胳膊就往外走。

如此大的动作牵动了江风的伤口,他嘴角一咧,抽了口气,然后瞪着叶安苒说道:“将我放下,我自己能走。”江风语调似乎很强硬,但是他的气息早已虚弱,任谁都能听出来,现在的江风,已是强弩之末。

叶安苒没有去看江风。她眉头紧蹙,接了话茬说道:“你要真还有力气,刚才我替你诊脉的时候,你为什么乖乖坐在那里?你还是拿好你的剑,再帮我掌着火吧。这地方不能再呆了,我要先找人替你疗伤解毒。”

江风无言以对,更想不通这位叶二小姐葫芦里到底卖的什么药,当下也只有先依着她,再找机会离开。随即,江风紧了紧阆风剑,又以自己的配剑作为支撑,随着叶安苒一步步往外走。

叶安苒一边带着江风向门口挪动,一边没好气地说道:“江道长,有件事咱们提前说好,这次你若能逃离追捕,我也算是对你有恩的人了。下次我不允许你再拿剑指我,否则——”

叶安苒的话还未说完,门就突然开了。江风猛一抬头,正好跟来人四目相对,惊道:“师妹,是你?”

来人正是阆风台玉灵真人的女儿,陆秋初。此时的她依旧披着那件不合身的黑衣,放下了高耸的兜帽,略显稚嫩的脸颊仍然泛着微红,手里依旧紧握着那柄削铁如泥的宝剑。

“师兄!”看到江风满身伤口,陆秋初倒吸了一口冷气,一时间竟不知所措,急得快哭了出来,“怎么会……师兄怎么会伤得这么重?”

江风勉强笑了笑,安慰道:“师兄没事的,别哭。”

叶安苒也说道:“小秋,现在不是哭的时候。我们还是先离开这里,万一有更多的人来,我们就插翅难飞了。对了,我的剑你先拿着,再把我那件衣服给你师哥披上。”

陆秋初点了点头,又将黑衣披在江风身上,“师兄,外面湿气重,你先把它披上吧。”

说罢,江风披上那件黑衣,被陆秋初与叶安苒一同扶着离开了破庙。月光虽弱却足以视物,然而三人行进缓慢,叶安苒心中焦虑,“我没有武功,小秋也只能对付一般人,万一一会儿再有人追来可怎么办……”然而走了许久,三人都没有遇到追捕,叶安苒深感奇怪,便问陆秋初:“小秋,那些捕快到现在都没追过来,是不是有点奇怪?”

“安苒姐姐,那些捕快逃去迷阵的方向了,估计这时候还在里面绕呢。”陆秋初略骄傲地回复着,“那是以前爹爹摆给我玩的,很少有人知道。我没想到刚才那些人竟然冲着迷阵方向逃,想喊住他们的时候已经来不及了……我想今晚他们不可能逃出来,除非他们能碰到阆风台的人……”陆秋初的声音越来越小,情绪也越来越低落。

听着陆秋初的话,叶安苒心里也很难受。阆风台毕竟是小秋的家,如今遭此巨变,不知何为归处。想到这里,叶安苒突然说道:“小秋,我们回城里吧。”

“为何?”回应的却是江风。

叶安苒认真回复道:“你重伤在身,可我们没有药来给你疗伤;就算想逃到其他地方,以你现在的情况也走不了多远。倒不如想办法进城,换些钱,买点药,再雇辆车。”

听罢,江风若有所思。叶安苒继续说道:“放心,既然救了你,我就不会让你落到旁人手里。小秋,我们走快点。”陆秋初应了一声,便按照叶安苒的吩咐做了。

三人走出树林,到了城外村庄。此时天还未亮,叶安苒眉间紧蹙,汗水已经顺着鬓角滴下,体力怕是到了极限。

突然,叶安苒做了个噤声的手势,三人停在了一户农家的院墙旁。

“小秋,你去这家看看能不能拿几套衣服,你们俩都穿着阆风台的弟子服,太扎眼了。我和你师哥继续往前走,你一会儿追上来。”

陆秋初向叶安苒一点头,脚下轻点,飞上墙头,又稳稳地翻入院中。江风心中暗喜:“师妹的轻功倒又精进了。”

眼见陆秋初翻进院内,叶安苒略做停留,抹了下汗,又深吸一口气,继续架着江风往前走去。两人向前走了一小段路,陆秋初便赶了上来,手里多了三套衣服。

“真巧,那家里竟然没人,院子里面还挂着晾晒的衣服,没收回屋里。”陆秋初兴高采烈的心情溢于言表。

“哈,那倒真是幸运了。前面找个隐蔽的地方,咱们把衣服换好,再想办法进城。”叶安苒已是气喘吁吁。

这时候陆秋初才意识到,叶安苒——这个从小锦衣玉食的大小姐——在没有丝毫武功的情况下,竟然拖着江风走了这么久,而且一路上都没有喊累。她心中不忍,说道:“安苒姐姐,你先歇一会儿吧,我来扶师哥。”

江风叹了口气,对叶安苒说道:“把我放下来,我自己能走。”

叶安苒不以为然,“算了吧,你要是能走,我这一路上又何必这么累?反倒是小秋你,要好好保存体力,要是真碰到了敌人,就只能靠你了。这天也马上就亮了,咱们必须尽快赶到城门,若是那群捕快已经回了衙门,只怕城门会更难进。”

这时只听得从上方传来一声轻笑:“叶二小姐心思缜密,若是阆风台上下有你一半机敏,也不至于落得如此下场。”

这一声清亮的嗓音似是凭空而来,三人俱是惊悚。但见一高挺青年立于屋顶之上,俯视着狼狈的三人。

陆秋初正待拔剑上前时,江风低声制止了她。他深知,此人轻功远在自己之上,陆秋初贸然前去,决计讨不到任何便宜。他定了定神,强撑精神朝青年拱手施礼,开口道:“敢问阁下尊姓大名。”

“唐玖言。”青年话中带笑,似是早已看穿一切。

忽然刮来一阵凉风,吹起青年的衣袂。待风吹过,青年继续道:“三位不冷吗?在下的马车已停在西边村口,还请三位赏光。”

话音刚落,唐玖言便消失得无影无踪。

叶安苒身上已经惊出了冷汗,她长长吐出一口气,抿抿嘴,道:“刚刚那人提到了阆风台,不知到底什么来头,我们怎么办?”

“咱们别去了吧……”陆秋初小声道,“那个人看起来有点奇怪……”

江风思忖少倾,道:“我要过去。”

“好吧。”叶安苒不再言语,只重新架起江风,往西走去。

走到西边村口,三人便看到马车停在路中央。这马车宽阔高大,比起寻常马车更显气派。然而比起马车,更让人称奇的是车前的两匹马,它们通身雪白,唯四蹄漆黑,双耳如筒,目光灼灼。

叶安苒脱口道:“竟然是螣雾……”

陆秋初好奇地问道:“安苒姐姐,你说什么?”

叶安苒指着那马答道:“它们是前朝时由塞北传入我国的宝马,名叫‘螣雾’。相传日行千里、夜行八百,却极难饲养。我家曾以千金求得一匹,并派了经验丰富的马夫去照顾,可惜最后没有成功,来年开春时那马就死了。这人竟能拿两匹来拉车,真是匪夷所思。他到底什么来头……”

正说话间,立于车前的车夫朝他们走来。他头戴竹编斗笠,身着粗布麻衣,脚踩多耳麻鞋,周身只似平常人,然而步伐稳健,足下生风,想来也不是寻常之辈。

江风见状,低声吩咐道:“一会儿这人过来,我便与他前去拜会唐玖言。你们二人在此等候,若有异样,立即离开。”