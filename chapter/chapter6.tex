% !TeX root=../main.tex
\chapter{消息}
这一天,城中街市热热闹闹。人们熙熙攘攘,或为生活奔忙,或是寻欢作乐。

正所谓风尘之地必有性情中人。只见一个十四五岁的少年,一身锦绣袍,头戴鎏金箍,正侧身坐在酒楼二层的窗边,一边喝酒,一边听外面说书。

“这相传呐,阆风派第六代掌门与我朝太祖颇有渊源。是以自中原一统以来,阆风一脉便被奉为国教。数十年来,阆风逐步在武林中一家独大。就连咱们普通百姓,也多信奉他家的真武帝君。大伙儿说对不对?“

说书人故意顿了顿,等着听客们的反应,反应最大的人里,自然少不了说书人找好的托儿。

“可惜呐,这一夜之间,阆风台竟遭变故。掌门身死,他徒弟竟然就是凶手。啧啧,看来这阆风派教徒无方啊。“

听到这里,听众们又不由得叹气,其中不乏有咒骂之声。

少年冷笑,继续默不作声地听着。

说书人继续说道:“到今天,阆风台已经遇袭五天了,消息早已传遍了整个武林。平日里啊,武林各派无不盼望结交阆风派;现如今,各门各派虽不明说,但都想着取而代之。更有甚者,当初被阆风台拒之门外,现在或者避之不及,或者暗地奚落。唉,人心不古啊。”

听到这里,少年简直快笑得掉下去了,“这说书的真是两头堵。刚才还在背后说人家教徒无方,现在就惺惺作态同情起来了。”

正想着继续听书,少年突然察觉,在另外一边的楼里,自己跟着的一位熟人正急匆匆地下楼离去。“清远师兄这么快就聊完了?”

想到这,少年也就顾不得其他。他一跃而下,追着那人而去。

少年追了许久,最终来到了一处颇大的宅子,正门处的匾额写着”扶英剑庄“四个大字。

“清远师兄果然回家来了。”

然而少年并未从正门进入。他跑到一处高墙之下,倒退几步,急奔之后接力踩上墙面,另一只手攀上高墙用力一捞,凌空一个漂亮的转身,稳稳立在了墙内。

说来也怪,叶清远回来之后,并没有什么进一步的举动,反而跟平常一样,去了练剑坪指导几个师弟妹的剑法,又跟几个入门较早的师弟说了会儿话。

少年觉得这太不寻常。

”清远师兄急急忙忙回来,却什么都不做?哼,我倒要看看你葫芦里卖的什么药。”

就这样,少年整个白天都待在宅中。这对他而言实在太不容易。往常这个时候,他都在外面疯跑。不过一想到自己要盯着叶清远,他也只好忍耐了。

夜幕降临,叶清远从弟子房中蹑手蹑脚地出来,径直向庄主书房而去。一直盯着他的少年也偷偷跟了上去。

夜色正浓,书房里飘着昏暗的灯光。叶清远正跟庄主交谈:“探子回报,有人在霞云岭西北的清水镇见过二小姐。目前她身体无恙,但与她同行的男子伤得不轻。这几日,二小姐他们在镇外的清凉山上借宿。”

“男子……伤得不轻?”叶庄主眉头紧蹙,“江湖盛传,无为的大弟子江风盗取阆风剑,犯下欺师灭祖的勾当。官府已下了海捕文书,但至今一无所获。你刚才说的受伤男子,莫不是他?”

“属下不知,但当日于阆风台上失踪的人,除了二小姐,只有这个江风和另一个女弟子;而探子也说,二小姐是与一男一女同行,想来不是巧合。”

“若消息属实,江风重伤而苒儿却毫发无损,那阆风台……”叶庄主仿佛突然想到了什么,突然话锋一转:“挑十个武功最好的弟子,骑最快的马,连夜出发,把苒儿接回来。”

叶清远答应一声,正要转身离去,叶庄主又叫住了他:“不许告诉清流。记着,要暗中行事。”

“弟子明白。”

这时,窗外仿佛有什么动静。叶庄主朝窗外看去,只听得几声猫叫,走到窗边四下查看,也并没有发现有什么人。他关上窗户,又嘱咐了几句。

“呼——”躲在树干后的少年长舒了口气。他轻点一下,无声落在地面,朝着已经关闭的窗户轻蔑地一笑,转身就跑向自己的房间。

到了房间,他拿起自己的佩剑,仔细检查了一番。

这剑由玄铁所铸,长四尺五寸,宽二尺一寸,重六十余斤。这少年身材高大,想来也是天生神力方能挥舞得了这样的神兵。

检查完毕,少年对着剑说道:“切,庄主真是。明明师姐失踪了那么久,手下全部的势力也都放了出去。现在好不容易有点消息,还不告诉我?放眼整个山庄,有几个能比得过咱们俩?不让我去?我还偏要跟去。苍雪,咱们去找师姐!”

叶清流背着剑,吹息了烛火,就这么消失在黑暗中了。

第二天拂晓,叶清远带着一群人辞别了庄主,偷偷出发去寻找叶安苒。

阵阵马蹄,也迅速被清晨的宁静所取代,直到一声叫喊打破了这一切:“快去禀报师父,清流师弟又跑出去了!”

他们不知道,当叶清远带着一队人马出城时,已经在城外等候多时的叶清流也跟了上去。为了防止被发现,叶清流起初还在有意控制着彼此的距离,后来才发现自己的考量有些多余。

“混蛋。从外面搞来的马,果然不如家里的好。”

话分两头。为了避免过分引人注意,叶清远要求弟子们乔装打扮。每到一处城镇,他们便分散在几家不同的店内投宿。这样的安排虽然降低了暴露的风险,但不可避免地耽误了一些时间。

几天后,叶清远一行人到了清水镇。在镇内,他们四处打听清凉山的情况。可惜,除了听说清凉山上以往不曾有人居住外,他们没有任何其他的消息。

叶清远心中疑惑,既然镇上的居民都称清凉山上无人居住,那探子回报的消息又从何而来?更重要的是,几天来他已经感觉到有人仿佛在跟着自己,但又一直找不出这人是谁。为今之计,只能带着众弟子上山,亲自寻找一下了。

翌日,叶清远带领众弟子上清凉山。刚进山时,叶清远便注意到,在一条僻静山道有新近踩踏的痕迹。

一行人顺着痕迹一路前行,半途中,叶清远发觉一路上一直存在的被追踪感渐渐消失。他不敢怠慢这些许的变化,便提示众弟子小心应对周遭一切。

叶家众弟子平素训练有度,只见叶清远示警,便迅速打起精神。众人一边缓慢前进,一边探听四周动静。

过了许久,阳光直射下来,地上只留下斑驳树影。和风舒畅,虫鸣鸟飞,一片安详。

叶家众弟子一路上绷紧神经,已觉疲惫。一路上时有时无的线索,也不断敲打着他们的耐心。

终于,一座别样院落出现在他们面前。

在院落正门处,几个小厮正在打扫。旁边立着一根一面彩旗,旗上绣着一个大字——唐。

叶清远下马上前问道:“敢问你家主人如何称呼?”

小厮还未回话,一个声音便从身后传来,”想必几位是扶英剑庄的弟子吧。“

众人惊讶,急忙回身。只见来者气势不凡,举手投足间,不似山野村夫。

他见众人神情有异,便拱手施礼,却也露出了自己关节突出的双手。

“哈,忘了介绍,在下唐栗,就住在这座别苑当中。”

叶清远想到方才所见的旗帜,施礼问道:“想来阁下便是这别苑的主人。我等昨日到达山下城镇,镇中居民皆不曾提到山上有人居住。今日我等进山,原是顺着一些痕迹而行,不料竟到了阁下住处。不知阁下何时居住于此?又是否见过什么人?阁下又为何知道我等的身份?”

唐栗坦然一笑,回道:“叶公子误会了。我并非这别苑主人。我家公子前几日才搬来这里,但这之后,却并没有见到什么生人。至于其他的问题,叶公子可以当面向我家公子确认。”

叶清远闻言,心情有些低落。此次入山,原本为寻叶安苒而来,却不料两日来全无所获。他本欲就此告辞,但如今时间近晚,若此时下山,恐怕要在林中过夜,难免危险。无奈,只得请唐栗引荐别苑主人,希望能投宿一宿,明日再下山。

唐栗却并未引众人入院。

“叶公子,我等未得我家主人应允,不敢擅自带人进入。叶公子可自行询问我家公子,若得公子回应,我也好引众位前去。”

叶清远心知,这唐栗是要试试他的本事。想他行走江湖多年,倒第一次被人瞧不起。

此时也容不得多想。他面朝别苑正厅,运起丹田之气,以内力传声入室。

“扶英剑庄叶清远携众师弟前来拜会!还请唐家主人移步相见!”

一盏茶过后,一名女子自别苑而出。待她逐渐走近,叶清远征住了。

这……二小姐?!”